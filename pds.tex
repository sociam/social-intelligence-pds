
\documentclass[runningheads,a4paper]{llncs}

% \usepackage{amssymb}
\setcounter{tocdepth}{3}
\usepackage{graphicx}

\begin{document}

\title{The Future of Social is Personal: The Role of Personal Data Stores in Social Interfaces}

% by Max Van Kleek, Kieron O’Hara

\maketitle

\section{Introduction}

A key characteristic common to the various kinds of ``social intelligence'' described in this volume is one of enhanced autonomy through technological support.  Such autonomy allows constituents of a society to form new connections with others dynamically as needed, promoting a more adaptive, flexible and robust social fabric than those of traditional structures, in which efficiency leads a majority to rely on a handful of central, fixed intermediaries. This observation immediately prompts the question of whose interests that ``efficiency'' is benefiting - the intermediaries' or the users'.

While we see technology being applied in many ways to generalise the benefits and enhance the autonomy thus described, the storage of personal information is one area where it has, thus far, been used to drive a reversal in such trends, towards more centralisation. Currently, a handful of dominant platform vendors and application service providers are grappling for control over individuals' personal information archives, trying to accumulate as many users as possible before the others. This trend as business model began with the rise of so-called ``Web 2.0'', in which sites became sophisticated apps and content-management platforms designed to facilitate the creation and sharing of user-generated data and content; content that started with a few social network profiles and blog posts, but gradually grew to encompass the entirety of personal data people keep, from files and documents to film and music archives. Thus began a migration of personal digital artefacts off the individually-administered personal computers into various information spaces of the web.  The assimilation of personal data off personal digital devices has accelerated particularly recently, as Web application and service providers have started to create deep integrations with personal computing devices with examples such as Facebook Home\cite{facebook}, Windows Skydrive\cite{skydrive} and Apple's iCloud\cite{icloud}, respectively. Such services have extended the reach of Web services into the intimate digital spaces of one's personal digital devices, offering backup and management services for these private data collections as well.

What are the implications of this centralisation?  Although the ultimate, long-term implications of this shift are not yet fully understood, several immediate consequences are apparent. Fundamentally, the delegation of responsibility of managing one's personal information to third party service providers necessitates relinquishing control over various aspects of how these data are handled and controlled, ranging from how they are stored and represented, to how (and when) they can be accessed, as well as to whom access is granted.  When such third party delegation is done in the context of the increasingly pervasive business model of deriving sustaining revenue directly from these data themselves (through targeted advertising or licensing to third parties), platforms are essentially incentivised to collect from as many individuals as possible, and to create an experience or mechanism that further retains them as long as possible.

While this mechanism has thus far been hugely successful at creating extremely profitable services of the likes of Facebook, Twitter, Twitter and Google, the result has been an increasingly fragile ecosystem in which a majority of Web users have come to rely on a handful of service platforms, which are, in turn amassing a disproportionate quantity of users' personal information.  This centralisation has occurring not just for Web users from the United States, where most of these services are based, but internationally as well, raising concerns pertaining to each country's sovereign rights of access to data of its own versus other nations' citizens, as well as issues pertaining to compliance and enforcement of data protection laws across international boundaries \cite{internationalownership}. Moreover, the fact that these platforms are incentivised to get users to disclose as much of their information as possible has led to an artificial forced tradeoff between participation and privacy; in order to enjoy the most basic features of the Web, they have to \emph{give their data away}, thereby sacrificing control over their data and potentially their privacy.

This misalignment of incentives between \emph{what users want to do with their data} and \emph{what platform providers want to do with their data} has the potential to destructively interfere with development of context-sensitive applications that promise more effective, personalised, behavioural-adaptive applications that rely on richer and more sensitive data models, due to either actual or perceived privacy risks entailed.  Moreover, the dependency relationships that result from this process places unprecedented power in the hands of these companies, leaving individuals fundamentally powerless towards effectively switching to alternative providers in the long term; the result of this is an overall reduction of autonomy and mobility, potentially ultimately leading to increased fragility, fragmented data spaces and lost or forgotten data\cite{lostlegacy}.

However, a basic assumption that powers this dependence is the disparity between the data management capabilities held by the end-users of the Web from those that provide the hosting and storage. In this chapter, we question this ``thin client'' model of Web computing by examining an alternative approach that places the responsibility of data management back with the users who own it, but in a way that is natural and manageable, while supporting the same social, dynamic interaction flows they are used to on the Web.  This set of capabilities we refer to as \emph{personal data stores} (PDSes), the technical goal of which is to augment user computing devices with secure data storage, hosting, and sharing capabilities which can be used to longitudinally archive and manage valuable information, as they interact with one another and third parties respectively.  

To derive the requirements for personal needs for what such a platform through insights from the field of Personal Information Management (PIM).  Second, we present a brief summary of existing platforms being used to manage personal information and their characteristics.  The chapter concludes with a discussion of how these platforms may change the socio-economic landscape of the Web, and the ways personal data is shared, collected and handled.

\section{Background and Brief History}

The genesis of digital personal data archives actually pre-date the digital computer entirely, to Vannevar Bush's Memex vision of 1945\cite{memex}, which proposed a mechanical framework for supporting the collection, archiving, and organisation of information to facilitate later cross-reference and retrieval.  Douglas Engelbart's realisation of NLS\cite{engelartnls} in 1969 demonstrated many ideas that would not be realised in any commercially available products for the next decade, including one of the earliest graphical user interfaces, the computer mouse, drag and drop manipulation, dynamic hierarchies, hyperlinks, hotkeys multi-view representations, and real-time remote collaboration.  Finally, the introduction of the personal computer in 1984 was shortly followed by a many first generation personal information management tools for them, ranging from personal database systems like Filemaker \cite{filemaker}, to digital calendaring and contact management tools, to file managers, spreadsheets and word processors.

Computer science research in the 1990s investigated approaches of automatic sensing and capturing aspects of everyday life activities into personal \cite{lifelogs}, starting, perhaps with the Pepys Memory Prosthesis \cite{pepys}.  Wearable and ubiquitous computing research continued this line of investigation, pursuing method of capturing of higher-resolution and more complete logs of people's activities (e.g,. MyLifeBits \cite{mylifebits}), and applications for data-mining lifelogs for various important life patterns (e.g. Life Patterns \cite{clarksonphd}).  The next decade saw specific evaluations of lifelogging in various specialised contexts, including healthcare for chronic disease maintenance, including memory prosthesis applications for alzheimer's patients \cite{}, and cognitive behavioural therapy.

Simultaneously, the rapid rise of the Web brought an variety of apps and services for managing many kinds of information, ranging from the personal and sensitive to social to public.  With increasing quantities of the population ``going online'' emerged a market for the personal information people were putting online, along with concerns over privacy, security over one's personal data, and rights to access. Government initatives to give consumers more protection over various aspects of both how data about them could be collected and handled were proposed and trialed with modest success in the United States and and more success in Europe.  Simultaneously, independent research efforts in trying to give end users as consumers more control over their online privacy began to emerge such as the \emph{Vendor Relationship Management (VRM)}, which sought to not only investigate technical solutions but legal and economic frameworks that would lead to more beneficial outcomes for both consumers and businesses through consumer-empowerment \cite{vrm}.  Out of this work emerged the earliest mentions of Personal Data Stores, in the context of online e-commerce, which sparked from around 2011 more than a dozen different Personal Data Store offerings, platforms and services backed by commercial start-ups \cite{ctrlshift-report}. 

The potential impact of personal data store technology towards driving new models of e-commerce and new experineces for end-users has been the focus of substantial interest recently among independent research organisations. The World Economic Forum commissioned a report on the personal data economy and ways to ``unlock its value'', outlining a programme projecting Personal Data Stores to be a core enabling mechanism through which emerging personal-data rich applications could thrive while simultaneously respecting the privacy requirements of individuals online\cite{WEF-report}.  Similarly, independent research organisation Ctrl-Shift also led a comprehensive analysis on emerging Personal Data Store efforts and their roles in information markets from a socio-legal-technical persepctive \cite{ctrlshift}.  Complementing this in the UK was a government initiative called \emph{midata} \cite{midata} to give their customers direct and unfettered access to data kept about them by companies.  The success of \emph{midata} has been described to be contingent on several important steps, including realising effective tools such as personal data stores for letting indivdiual users easily consume, consolidate and make use of this data once it is made available.
  
Yet despite the extensive needs analysis and market potential identified, early personal data store offerings have thus far failed to attract substantial attention from users.  While a number of factors are likely responsible, so the lack of interest among users has been attributed to the fact that many of initial PDS platforms have sought to simply re-create exisiting end-user experiences offered by popular apps and Web platforms, rather than creating new functionality for users.  Despite the benefit that these PDS offerings provide in terms of data security, users are often less compelled to try something new if the tangible experience nothing new, while data security remains an abstract, inestimable threat which does not necessarily easily compel behaviour change \cite{abstract-threat}.  Finally, since the very purpose of PDS offerings is to protect user data from third party access, these platforms cannot derive revenue from user data and must resort to subscription models - which are always less attractive to new users than than offerings that are completely free to use. 

%This context thus has started to highlight some of the difficulties with introducing new PDS approaches within the existing, highly competitive and lucrative Web service and platform environment which has been powered by the economy of personal data.  This drove us seek 	

The difficulties that this community has encountered have led us to reconsider, from the ground up, the need(s) these platforms are meant to address, so that they can be used to design a platform that will fulfill needs beyond securely storing data , towards new applications that promote the more effective use of them in both personal and social contexts.  We first seek to establish a clear definition for PDSes based on a characterisation of what they were meant to achieve.  Second, we derive a requirements analysis based on the abstract definition, deriving insights from the personal information management (PIM) research community.

\subsection{(Re-)Defining the Role of Personal Data Stores}

%We begin with an abstract characterisation of what a technical capability for achieving the aforementioned goal, adopting the term  \emph{personal data store} (PDS) from the Vendor Relationship Management (VRM) research community \cite{vrm}.  % (The use of PDS in the singular is not intended to imply such a system would have to be a monolithic system, but might be realised through several or many disparate systems providing support).

The goal of the personal data store fundamentally to give individuals ability to safely keep, and effectively use any of their data for as long as they need it, and to share their data as they wish with whom they wish. Thus, we propose the following definition:

\begin{quote}

	A personal data store is a platform or service that allows individuals to manage and maintain their digital information, artefacts and assets, longitudinally, fully, and self-sufficiently, so it may be used practically when and where it is needed, without relying on external third parties. 

\end{quote}

This description leaves undefined the kinds of activities that might constitute ``manage'', ``maintain'', ``control fully'' or ``use'' this information, nor even what kind(s) of information we are talking about.  In order to approach a requirements analysis, one must consider both questions \emph{what} and \emph{how}; the kind(s), representation(s) of information to be stored and managed, and how the system is to support the user towards doing supporting use of and management of the data.  Toward this end, the fields of information science and human computer interaction (HCI), particularly the research field of Personal Information Management (PIM), have worked to document the ways individuals work with, and manage the information in their lives, both in personal and work contexts.  We thus propose that work on PDSes should be informed thoroughly by this literature, specifically in scoping \emph{what} PDSes might do and further \emph{how} they best do it.

\subsection{What Constitutes ``Personal Data?''}

The task of identifying all of the kinds of data a person might need to keep, manage and use is a complex and not easily scoped task.  Researchers in PIM have derived various working definitions of \emph{personal information} in order to effectively scope their field of study, and and have made progress towards potential functional classifications for kinds of personal information. One such classification by Jones et al. from \cite{kftf} is visible in Figure\ref{fig:jonestype}.

\begin{table}
\begin{tabular}{p{4cm} p{8cm}} 
Category & Examples \\
1. Owned/controlled by me & e.g., Email, files on our computers  \\
2. About me	& e.g., my credit/medical history, web history \\
3. Directed towards me & e.g., phone calls, drop ins, adverts, popups \\
4. Sent (provided) by me & e.g.,  Emails, tweets, published reports  \\
5. Experienced by me & e.g.,  Pages, papers, articles I’ve read \\
6. Relevant (useful) to me	& e.g.,  Somewhere ``out there'' is the perfect vacation, house, job, lifelong mate \\
\end{tabular}
\caption{Jones's 6 Types of Personal Information, from \cite{kftf}}
\label{fig:jonestype}
\end{table}

Jones takes an approach that distingushes among different kinds of information by how it relates to the individual in question; whether the individual experienced it, kept it, sent it, or received it, or whether this information refers to the individual or his or her activities.  The categories \emph{About me} and \emph{Relevant to me} are controversial because these definitions do not require individuals to be aware of the existence of the information; it thus establishes a sphere that goes beyond the scope of information experienced by the user.  We discuss the potential implications of including such information within the scope of PDSes in \emph{attentional challenges}. 

\subsection{Activities Around Personal Information}

Each person can access, use and manage information in many different ways throughout their everyday activities.  Moreover, there is considerable variation among the ways that different people manage their information, as documented in studies of people's office and home information environments for nearly a half century \cite{filerspilers}.  As a result, it has been relatively difficult to come up with a single characterisation encompassing all of these activities; several classifications have been proposed.  Returning to the PIM literature, Jones et al. propose a categorisation centering about a distinction between finding, keeping, and a set of ``M-level activities'', which encompasses managing and organising information archives (Figure \ref{fig:pimactivities}) \cite{kftf}. Jones Whittaker et al's slightly different categorisation, meanwhile, simply identifies 3 classes: keeping, management, and what he called ``exploitation'', as follows:

\begin{table}
\begin{center}
\begin{tabular}{p{4.5cm} | p{4.5cm}} 
Jones \cite{jones}, Jones and Teevan \cite{jonesteevan}& Whittaker et al \cite{whittaker}\\
\hline
(Re-)Finding &  \\
Keeping & Keeping \\
Meta-level activities (managing, maintaining) & Management \\
 & Exploitation \\
\end{tabular}
\caption{Jones and Teevan vs Whittaker's categories of PIM activities}
\label{fig:pimactivities}
\end{center}
\end{table}

Jones's classification introduces \emph{finding} as a primary activity that people perform; his definition spans a set of common behaviours including discovery \cite{}, information foraging\cite{}, orienteering\cite{}, searching\cite{} among other related behaviours in which people purposefully seek information or serendipitously encounter it in the course of other information activities.  Once this information is found, information is either consumed and internalised, or kept in an external archive, or bddoth, and this process of saving information externally is referred to as \emph{keeping}.  Beyond this activity of archiving, individuals might return to their archives to organise them, update them, remove entries that have become unnecessary, and so forth; such activities are referred to as the \emph{M-level} or \emph{Management} activities above.   Whittaker then includes a fourth behaviour, \emph{exploitation} which he uses to abstractly refer to the ways in which the information is then used.

Among such uses, while the foremost might be to \emph{inform} an individual towards making a decision, many other uses of information also exist.  For example, information might be created for the explicit purpose of \emph{reminding} a person of past or future events, activities or details. Other purposes might be to \emph{measure} and keep track of the time-evolution of some phenomenon so that it can be easily understood.  When this measurement is about the individual's own activities, the purpose might be for providing \emph{feedback}, which may be vital for domains such as cognitive behavioural therapy (CBT)-like programmes.  This feedback may, in turn, along with other information, collectively serve to \emph{motivate} further activity or behaviour.  Finally, information may serve the purpose of \emph{external cognition}, in which information is created or manipulated for the purpose of facilitating \emph{understanding} or \emph{problem solving}.  This set of activities is often referred to as \emph{sensemaking} \cite{pirolli2005sensemaking}.

\section{Supporting Information Activities}

Technological support for each of these information activities has demonstrated the potential to change not only how they are conducted, but the contexts in which they are applied.  One salient example is that of Web search engines, a tool originally created for Web page information retrieval, but which has become a nearly ubiquitous tool for accomplishing tasks across a much broader variety of activities, spanning both desktop and mobile.   Another area is in supporting longitudinal keeping behaviours; tools that automatically perform off-site, incremental, and continuous backup such as Apple's \emph{Time Machine} \footnote{Time Machine - \url{www.apple.com/uk/support/timemachine/‎}} have become commonplace, allowing end-users to make their stored data more resilient to accidental deletion or data loss.

Yet such technological support has remained rudimentary towards most of the other aforementioned personal information activities, including reminding, sensemaking, discovery and orienteering.  Reminding in PIM tools, for example, has until only recently been limited to clock/calendar-based alarms that need to be explicitly set for a specific date and time, despite the rich variety of ``off-line'' strategies people have naturally adopted for their own uses\cite{belottitodo}.  While the basic calendar alarm remains heavily used, its precision, brittleness and intrusiveness have been documented to result in their loss of effectiveness, sometimes through extended  ``snooze wars'', in which users repeatedly dismiss alarms, resulting in their piling up over time.  Such alarms in such cases end up a burdensome annoyance, instead of the assistance they were intended to provide.  

\section{Challenges Distinct to PDSes}

The goal of providing individuals with the capacity to longitudinally maintain their own personal information imposes a number of unique challenges towards supporting the kinds of information activities just described.  In particular, are four uniqu sets of challenges that must be met; the first, most fundamental of which pertains to effective \emph{longitudinal keeping}.  Enabling individuals to store safely keep their data for a long time, while ensuring its continued accessibility and usefuleness impacts both the data formats and methods used to store them.  For example, since a person's physical computational hardware is likely to fail with age, methods need to be in place for ensuring robustness to such failures, such as multi-device replication and the ability for data to be easily migrated from older to new devices over time.   Moreover, as evidenced by Moore's law \cite{gray2000rules}, since the technical capabilities and properties of such data storage devices and platforms are likely to fundamentally change over time, PDSes must be designed to be able to accomodate (and take advantage of) such changes as they arise.

A second challenge concerns allowing individuals who might have little or no experience in the intricacies of data management to cope with the burden of data security and longitudinal maintenance.  Using current tools and services, for example, managing your data yourself still means taking pains to ensure that one's personal data is not lost to hardware and software failure, malicious attacks, or safely migrated to new platforms and devices; such efforts require vast investments of time, effort and expertise.  A general lack of expertise or willingness to do this  means that people currently rarely know how or bother with backing up or consolidating their data. Thus it is no surprise that individuals have been motivated to outsource maintenance of their data to third parties, such as cloud data providers we describe later.  In order to facilitate autonomy from such services, thus, PDSes must seek to directly support, and automate where possible, tedious data maintenance tasks that have plagued PC users for decades.  Such automation could both ensure compliance for promoting data security and integrity, such as continuous backup regimes, thereby countering recent studies of the extremely low compliance 

A separate set of challenges arise from the shift back from service-provider controlled data storage to a user-centered model of data management. As mentioned earlier, this transition will re-empower users to control the organisation of their data spaces, rather than having it dictated by third parties.  A second advantage to this approach is that it may eliminate the pervasive problem of data fragmentation \cite{jones}, by allowing individuals to keep consolidated, definitive copies of their information, instead of being required to distributing information among separate services by their types \cite{web-fragmentation}.  However, the challenge with the increased flexibility that this approach affords is that it requires re-consideration of how third-party applications and services can interact with such data, which have traditionally been pre-defined to operate on a fixed, typically application-provider established, set of data representation(s) and manipulations.  In a consolidated, user-centric data model, on the other hand, such representations may be be specified or modified by the individual, or by some other third-party application(s) on behalf of them, and thus applications themselves must be designed to accommodate such variability among representations.  We discuss how semantic data representations have been used to address such challenges later in approaches.

Perhaps the ultimate set of challenges, however, pertain to accommodating change as it affects both the information itself and the practices and activities surrounding it, over the years that a PDSes is intended to operate.  Technologies that bring in new ways that data is used and generated seem to be introduced every quarter, placing new demands how this information needs to be accessed, created and used.  The most recent examples include wearable computing and ``always on'' wearable sensor technology, from simple devices such as Fitbits \footnote{Fitbits - \url{www.fitbit.com}} and Fuelbands\cite{Nike+ Fuelband - \url{www.nike.com/fuelband}}} that unobtrusively but nearly constantly measure simple aspects of an individual's activity, to complex computational devices that can both deliver and capture information in high fidelity and quantity anywhere, such as Google Glass\footnote{Google Glass - \url{www.google.com/glass}}.  Such devices, as well as innovative new apps in can in some cases bring about changes in norms pertaining to people's activities, including the ways people think about technologies themselves.

Looking forward at some of the ways such technologies might impact information activities, some have looked at the possible consequences and implications that ever-increasing information capture and access might have on the kinds of activities mentioned above.  While Bell and Gemmel have argued \cite{totalrecall} that such increased capture and access could create near-perfect records of our daily lives, allowing people to examine with unprecedented scrutiny their everyday activities, others such as Blanchette have argued that such a utopian views overlooks a great number of potential consequences other factors \cite{blanchette}. 

% information is created and shared, and new ways that society responds to and uses information for data robustness against physical device failure or loss, this requires PDSes to fundamentally allow people to freely replicate their or ensuring continued readability and usabililty as the hardware and software platforms that might be used with it changes.  The second is essentially to provide flexibility to adapt to future needs nd applications as they arise, so that PDSes continue to be useful as time passes.  A effort at doing The second is in self-maintenance; is the consideration of factors involved in the long-term maintenance and evolution of personal data collections.

 % suggests  towards more adaptive reminding strategies (e.g., with task based-reminding \cite{task-based-reminding}) have been demonstrated in a research setting, they have yet to be realised in a sufficiently robust context for widespread use.
% In discussing the support for PIM activities in the longitudinal context of PDSes, 

\section{Survey of Online Data Platforms and Services}

Given this characterisation of the various kinds of \emph{personal data} and activities around it, we can identify the ways that current online services fulfil the needs towards people's information types and activities.

Figure \ref{fig:cloudstorage} characterises the top five personal data cloud platorms by number of users. While Facebook may not be considered and end-user personal data storage provider of the likes of Dropbox, it remains one of the world's largest brokers of personal information.  Of particular interest is its introduction of Timeline in December 2011, when it started encouraging users to document the entire chronnology of their lives on the service, prompting users to backfill information about their lives from before they joined the platform through specific questions and prompts.  As a result, Facebook has quickly amassed one of the world's largest single collections of lifetime biographical information directly elicited from individuals.

Facebook only the supports the storage of very specific information forms, spanning status updates, likes, photos, messages to individuals and so forth.  While Google Apps and iCloud support similar structured data entries such as calendar entries, all but iCloud support general file storage.  A survey of why people used these storage services revealed that while backup had previously been the main reason for using online cloud services, multi-device access and sharing/collaboration have quickly eclipsed backup for reasons people use such services online \cite{listitstudy}. The primary use of Facebook, meanwhile is to stay connected with others, as well as several emotional reasons, spanning reasons of self-actualisation and to fulfill the need to belong \cite{why-do-people-facebook}.

\begin{table}
\begin{tabular}{p{2.2cm} p{8cm} l}

\emph{Facebook} & Profile incl. Timeline; Friends; Events; Group memberships; Biographical history; States favourites; Preferences; Message archives; Liked pages, images, products; Places visited. & Free \\

\emph{Google Apps and GDrive} & Any files; Google Docs; calendar; G+ profile; identify and profiles of friends; search history; page access history; bookmarks; locations visited & Freemium\\

\emph{Apple iCloud} & iWork Documents, Photos, Calendars, Passwords (Keychain) & Freemium \\
Dropbox & Any files & Freemium \\

\emph{Skydrive} & Office Documents; Any files. & Freemium \\

\end{tabular}
\caption{Commercial third-party cloud storage offerings}
\label{fig:cloudstorage}
\end{table}

However, these services primarily pertain to the management of a fraction of the personal data encompassed by Jones's definition above, specifically ``data owned/controlled by me''.  If we also extend consideration to online services that host and collect ``data about me'' as well, there are now an increasing number of sensor-driven apps and services that facilitate the tracking of various, routine aspects of everyday life activities, spanning purchases, movements, wellbeing vital statistics; we list such life tracking sites in Figure \ref{aboutme}.

\begin{table}
\begin{tabular}{p{2cm} p{6.1cm} p{2.54cm}}
Service & Description & Logging Method \\
\hline
Foursquare & Visits made to points of interest & Manual check-ins \\
\hline
Moves & Complete history of a person's movements throughout the day as recorded from smartphone app  & Sensed via smartphone app \\ 
\hline
Mint & Access to personal banking records (tracking spending) & Automatic \\
\hline
Withings; Runkeeper & Access to weight, blood pressure, heart rate & Semi-automatic\\
\hline
Fitbit; Fuelband; Jawbone & Daily activity levels & Sensed via worn sensor \\ 
\hline
Wattvision; Stepgreen & Energy consumption & Automatic (service provider) \\
\hline
Moodpanda; Mappiness; Gotafeeling & Mood & Experience Sampled \\
\hline
CalorieCounter; Fooducate & Daily calorie consumption & Manual \\
\end{tabular}
\caption{Web lifelogging services that facilitate the capture and logging of everyday life experiences. }
\label{fig:aboutme}
\end{table}

While both categories of services broker significant amounts of data, these do not generally meet the requirements for personal data stores, as service providers ultimately control how this data is stored, secured, and have full access to its contents.  Other services, meanwhile have been launched ofcused on security of user data; a list of such services are listed in \ref{fig:pdsofferings} and are sometimes referred to as the first generation of ``personal data store'' offerings.

\begin{table}
\begin{tabular}{l p{8.5cm}}
Personal.com & Cloud svc for keeping important structured data of specific schema types (passwords, contact details)  \\
Mydex & Cloud svc centered around specific structured data and identity verification  \\
\end{tabular}
\caption{Personal Data Store offerings which encrypt data to provide a high degree of user data security, e.g., only the user has access.}
\label{fig:pdsofferingsp}
\end{table}

\begin{table}
\begin{tabular}{l p{9cm} }
aerofs & Commercial solution for self-hosting a centralised dropbox-like service \\
bittorrent sync & Commercial peer to peer file synchronisation software for personal computers \\
gitannex &  FOSS Distributed file metadata maintenance system for advanced users \\ 
cosicloud & FOSS self-hosted cloud platform for plug computers offering mail, photo, contact and metadata hosting and storage \\
data.fm & FOSS RDF-based Web data store with linked data support
\end{tabular}
\caption{Self hosted personal data platforms}
\label{fig:selfhosted}
\end{table}

% \begin{table}
% \begin{tabular}{l p{5cm} p{7cm} l l l } 
% Service & self-hosted & open-source & data types & security & longevity \\
% Dropbox & n & n & gf & sp & n/a \\
% bittorrent sync & y & n &  \\
% gitannex & y & y & n \\
% cosicloud & y & y & sft/s & n & n \\
% SugarSync & \\
% WD MyCloud & 
% \end{tabular}
% \caption{Self-hosted personal data storage platforms for end-users}
% \label{fig:se}
%  \end{table}

% While personal backup appliances such as ioSafe, WD My CLoud, and Apple's Time Machine have aimed at providing self-hosted data management encapsulated in simple ``plug and play'' data appliances for the home, adoption of even rudimentary home backup solutions remains low. Although current estimates are poor, it is thought that 10\% of individuals effectively regularly back up their home computers, with the majority backing up only ``periodically''.

% \begin{table}
% \begin{tabular}{l p{5cm} p{7cm} l l l } 
% ioSafe & \\
% WD My Cloud & \\
% Time Machine & \\
% \end{tabular}
% \caption{Personal backup devices}
% \label{fig:se}
% \end{table}

\section{Technical Approaches}

\subsection{Proactive support: Context-sensitivity and automation}
Location based reminding apps, such as Checkmark \footnote{Checkmark - \url{itunes.apple.com/us/app/id524873453?mt=8}} have started to break out of aforementioned calendar-alarm to more adaptive reminding, by allowing alarms to be set sensitive to the user's automatically-sensed physical location.  While still very simple and potentially fallible, this approach highlights the potential for greater context-sensitive support for all of the various personal information activities.  

Context-sensitive support is particularly attractive towards future PDS work as it seeks to apply automation to provide

\subsection{Semantic Technology}

% \subsection{Towards a Requirements Analysis}

% The next challenge, thus, is to identify how the design of future PDSes might be informed by such studies and abstract characterisations of the kinds of information people manage and the ways they go about doing so.  

% This is challenging for a number of reasons; first, the above characterisations are considerably high-level.  Although examples can be generally identified for each, it may be difficult to extrapolate from specific examples a set of scoped design recommendations.

% A second problem arises from the goal of long-term data management; since information can remain valuable for years or even generations, a central purpose of PDSes is to support the keeping and use of data over such durations.  While merely storing data for such timespans is challenging given the rapid pace of digital infrastructures, data formats and computing platforms, supporting its effective use over such a duration may be considerably more difficult, requiring consideration of how not only the information might change but its use as well.

\section{INDX: A Prototype Personal Data Store}

\subsection{Architecture}

\subsection{Implementation}

\section{Discussion}

\subsection{Ownership and Value}

\subsection{Data Licensing and Rights Management}

\subsection{Time-resilient Data Formats}

\subsection{Data Literacy}

\section{Conclusion}


% Identifying the particular information forms to PDSes at any one moment involves enumerating all of the information forms commonly used, but also any that might be used in the future. 

% This definition of information form relies 




% \section{Scenarios: Benefits of PDSes}
% \subsection{Consolidation rather than Fragmentation}
% \begin{quote}
% Anna has just two new year's resolutions this year: to start to save money towards her daughter's education, and to lose some weight as recommended by her physician.  To track her progress towards these goals, she uses many online tools; her credit and debit cards provide itemised descriptions of every purchase she makes, while her mobile phone tracks her movements, including any exercise she she performs, or whether she drives or takes public transport to work.
% \end{quote}
% \subsection{Preservation and Inter-generational Access}
% \subsection{Context-Aware Computing}
% access the family archive
% URL and a secret key
% seamless integration
%   diet
%   activities : exercise, sleep
%   self-diagnosis
%   energy saving
% societal interaction



\bibliographystyle{abbrv}
\bibliography{pds}

\end{document}


